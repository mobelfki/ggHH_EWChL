The calculation builds on the one presented in Ref.~\cite{Heinrich:2017kxx} and therefore will be described only briefly here. 

The leading order amplitude in the full theory and  all the
amplitudes in the $m_t\to\infty$ limit were implemented analytically, whereas the
one-loop real radiation contribution and the two-loop virtual
amplitudes in the full SM rely on numerical or semi-numerical
codes.

The real radiation matrix elements in the full SM were implemented
using the interface~\cite{Luisoni:2013cuh}
between \gosam~\cite{Cullen:2011ac,Cullen:2014yla} and
the \powhegbox~\cite{Nason:2004rx,Frixione:2007vw,Alioli:2010xd}, modified
accordingly to compute the real corrections instead of the virtual
ones. The one-loop real amplitudes we generated with the version
2.0 of \gosam{}~\cite{Cullen:2014yla}, that
uses {\tt Qgraf}~\cite{Nogueira:1991ex}, \form~\cite{Kuipers:2012rf} and
{\tt spinney}~\cite{Cullen:2010jv} for the generation of the Feynman
diagrams, and offers a choice from {\tt
Samurai}~\cite{Mastrolia:2010nb,vanDeurzen:2013pja}, {\tt
golem95C}~\cite{Binoth:2008uq,Cullen:2011kv,Guillet:2013msa}
and \ninja{}~\cite{Peraro:2014cba} for the
reduction.  At run time the amplitudes were computed using
\ninja{}~\cite{Peraro:2014cba} and \avholo{}~\cite{vanHameren:2010cp}
for the evaluation of the scalar one-loop integrals.

For the virtual two-loop amplitudes, we have used the results of the
calculation presented in Refs.~\cite{Borowka:2016ehy,Borowka:2016ypz},
which used also {\sc Reduze}\,2~\cite{vonManteuffel:2012np} and {\sc
 SecDec}\,3~\cite{Borowka:2015mxa}.

The values for the Higgs boson and top quark masses have been set to
$m_h=125$\,GeV and $m_t=173$\,GeV, such that the two-loop amplitudes
only depend on two independent variables, the Mandelstam invariants
$\hat{s}$ and $\hat{t}$.  We have constructed a grid in these
variables together with an interpolation framework, such that an
external program can call the virtual two-loop amplitude at any phase space
point without having to do costly two-loop integrations.
We used the same setup for the grid as described in Ref.~\cite{Heinrich:2017kxx} and extended it in the following way:
We can write the squared matrix element as a polynomial of degree two in $\lambda$, 
\begin{align}
M_\lambda& \equiv |{\cal M}_\lambda|^2=A+B\,\lambda + C\,\lambda^2\;. \label{eq-amplambdadep}
\end{align}
Therefore it is sufficient to know the amplitude at three different values of $\lambda$ in order to reconstruct the full $\lambda$-dependence. 
Choosing $\lambda=-1,0,1$ we obtain
\begin{align}
A&=M_0\;,\; B=(M_1-M_{-1})/2\;,\; C=(M_1+M_{-1})/2-M_0\;.
\end{align}
In practice we used the representation 
\begin{align}
M_\lambda &=M_0\,(1-\lambda^2)+\frac{M_1}{2}\,(\lambda+\lambda^2) + \frac{M_{-1}}{2}\,(-\lambda+\lambda^2)\;
\end{align}
in order to get a more straightforward uncertainty estimate.

In fact, to any order in QCD, we can separate the matrix element into a 
piece that depends only on the top quark Yukawa coupling $y_t$ (``box diagrams'') and a 
piece that depends on the Higgs boson tri-linear self coupling $\lambda$ (``triangle diagrams''):
\begin{align}
{\cal M} = y_t^2 {\cal M}_B + y_t \lambda {\cal M}_T.
\end{align}
The squared amplitude at each order can then be written as
\begin{align}
|{\cal M}|^2 = y_t^4 \left[ {\cal M}_B {\cal M}_B^* + \frac{\lambda}{y_t} ({\cal M}_B {\cal M}_T^* + {\cal M}_T {\cal M}_B^* ) +  \frac{\lambda^2}{y_t^2} {\cal M}_T {\cal M}_T^*  \right].
\end{align}
The above parametrisation makes it clear that the dependence of the cross section on
both the Yukawa coupling and the Higgs boson self coupling can be reconstructed
from only the 3 terms present in Eq.~\ref{eq-amplambdadep}.

In order to allow for comparisons and cross checks, we implemented
both the $m_t\to\infty$ limit as well as the full SM amplitudes at
NLO. This allows to run the code in four different modes by changing
the flag {\tt mtdep} in the \powhegbox{} run card. The possible
choices and the corresponding calculation, as presented in the
previous section, are the following:
\begin{description}
 \item[{\tt mtdep=0}:]{computation using basic HEFT,}
 \item[{\tt mtdep=1}:]{computation using Born-improved HEFT,}
 \item[{\tt mtdep=2}:]{computation in the approximation \ftapprox (full
   mass dependence in the Born and in the real radiation, Born-improved HEFT
   for the virtual part),}
 \item[{\tt mtdep=3}:]{computation in the full SM.}
\end{description}

\vspace*{1cm}


{\bf ToDo:} assess uncertainty related to the grid produced from SM PS points, but used for $\chhh$ values which populate the phase space very differently.
