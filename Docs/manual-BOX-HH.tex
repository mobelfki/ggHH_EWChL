\documentclass[paper]{JHEP3}
\usepackage{amsmath,amssymb,enumerate,url}
\usepackage{xspace}
\bibliographystyle{JHEP}

%%%%%%%%%% Start TeXmacs macros
\newcommand{\tmtextit}[1]{{\itshape{#1}}}
\newcommand{\tmtexttt}[1]{{\ttfamily{#1}}}
\newenvironment{enumeratenumeric}{\begin{enumerate}[1.] }{\end{enumerate}}
\newcommand\sss{\mathchoice%
{\displaystyle}%
{\scriptstyle}%
{\scriptscriptstyle}%
{\scriptscriptstyle}%
}

\newcommand\as{\alpha_{\sss\rm S}}
\newcommand\POWHEG{{\tt POWHEG}}
\newcommand\POWHEGBOX{{\tt POWHEG-BOX}}
\newcommand\POWHEGBOXV{{\tt POWHEG-BOX-V2}}
\newcommand\POWHEGBOXRES{{\tt POWHEG-BOX-RES}}
\newcommand\HERWIG{{\tt HERWIG}}
\newcommand\PYTHIA{{\tt PYTHIA}}
\newcommand\MCatNLO{{\tt MC@NLO}}
\newcommand\ftapprox{FT$_{\mathrm{approx}}$\xspace}

\newcommand\kt{k_{\sss\rm T}}
\newcommand\pthh{\ensuremath{p_{T}^{\mathrm{hh}}}\xspace}
\newcommand\pt{p_{\sss\rm T}}
\newcommand\LambdaQCD{\Lambda_{\scriptscriptstyle QCD}}
%%%%%%%%%% End TeXmacs macros


\title{The POWHEG BOX V2 user manual:\\
  Higgs boson pair production at NLO QCD with variations of the Higgs couplings} \vfill

\author{G.~Heinrich\\ 
  Max-Planck Institut f{\"u}r Physik, F\"ohringer 6, D-80805 Munich, Germany\\
  E-mail: \email{gudrun@mpp.mpg.de}
}

\author{S.~P.~Jones \\ 
Theoretical Physics Department, CERN, CH-1211 Geneva, Switzerland  \\
  E-mail: \email{s.jones@cern.ch}
}

\author{M.~Kerner \\ 
  Institut f{\"u}r Physik, Universit{\"a}t Z{\"u}rich, Winterthurerstr. 190, 8057 Z{\"u}rich, Switzerland\\
  E-mail: \email{kerner@mpp.mpg.de}
}

\author{G.~Luisoni\\
Max-Planck Institut f{\"u}r Physik, F\"ohringer 6, D-80805 Munich, Germany\\ 
  E-mail: \email{luisonig@gmail.com}
  }

\author{L.~Scyboz\\
 Max-Planck Institut f{\"u}r Physik, F\"ohringer 6, D-80805 Munich, Germany\\ 
  E-mail: \email{scyboz@mpp.mpg.de}
}
\vskip -0.5truecm

\keywords{Shower Monte Carlo, POWHEG, NLO, Higgs, EFT}

\abstract{This note documents the use of the package \POWHEGBOXV{} for
  Higgs boson pair production in gluon fusion. The code
  allows to produce NLO results matched to a parton shower for $HH$
  production with full top quark mass dependence, and also allows variations of the Higgs boson couplings.
  Other run modes of the program can be used to calculate approximations: the $m_t\to \infty$ limit (also called HEFT approximation),  the
  so-called Born-improved HEFT approximation, and  the \ftapprox
  scheme.}

\preprint{\today\\ {\tt POWHEG-BOX-V2}}

\begin{document}


\section{Introduction}

The \POWHEGBOX{} is a framework to implement NLO
calculations in shower Monte Carlo programs according to the \POWHEG{}
method. An explanation of the method and a discussion of how the code
is organized can be found in
Refs.~\cite{Nason:2004rx,Frixione:2007vw,Alioli:2010xd,Jezo:2015aia}.
The code is distributed according to the ``MCNET Guidelines for Event
Generator Authors and Users'' and can be found at the web page
%
\begin{center}
 \url{http://powhegbox.mib.infn.it}
\end{center}
%
In this manual, we describe the \POWHEG{} implementation of Higgs
boson pair production in gluon fusion at NLO QCD, as described in
Ref.~\cite{Heinrich:2017kxx}, based on the calculation reported in
Refs.~\cite{Borowka:2016ehy,Borowka:2016ypz}.  
The variations of the Higgs boson couplings are based on Refs.~\cite{Heinrich:2019bkc,Buchalla:2018yce}.
Please cite Refs.~\cite{Heinrich:2019bkc,Heinrich:2017kxx} if you use this program.

The code can be found under {\tt User-Processes-V2} in the {\tt ggHH} process-directory. 
An example input card ({\tt powheg.input-save}) and a run script ({\tt run.sh}) 
are provided in the {\tt testrun} folder accompanying the code.

This document describes the input parameters that are specific to the process. The
parameters that are common to all \POWHEGBOX{} processes can be found
in the manual {\tt V2-paper.pdf} in the \POWHEGBOXV{}/{\tt Docs} directory.

\section{Running modes}
The code contains the NLO QCD amplitudes for both, the full top quark mass dependence, 
for which $HH$ production is a loop-induced process, and the
$m_t\to \infty$ limit (sometimes called Higgs Effective Field Theory, HEFT), 
containing effective Higgs-gluon couplings, as the top-quark loops are integrated out.  
A detailed description of the different approximations can be found in Ref.~\cite{Borowka:2016ypz}.

This allows to run the code in four different modes, either by
changing the flag {\tt mtdep} in the \POWHEGBOX{} run card {\tt powheg.input-save}, or by using the script {\tt run.sh [mode]}.
The possible choices and the corresponding calculations are the following:
\begin{description}
 \item[{\tt mtdep=0}:]{computation using basic HEFT. All amplitudes
   are computed in the $m_t\to\infty$ limit.}
 \item[{\tt mtdep=1}:]{computation using Born-improved HEFT. In this
   approximation the fixed-order part is computed at NLO in the HEFT
   and reweighted pointwise in the phase-space by the LO matrix
   element with full mass dependence divided by the LO matrix
   element in HEFT. \textbf{This mode can only be run for variations of 
   the trilinear coupling $c_{hhh}$.}}
 \item[{\tt mtdep=2}:]{computation in the approximation \ftapprox. In
   this approximation the matrix elements for the Born and the real
   radiation contributions are computed with full top quark mass dependence
   (using matrix elements generated by {\tt
     GoSam}~\cite{Cullen:2014yla}), whereas the virtual part is
   computed as in the Born-improved HEFT. \textbf{This mode can only be run for variations of 
   the trilinear coupling $c_{hhh}$.}}
 \item[{\tt mtdep=3}:]{computation with full top quark mass dependence.}
\end{description}

Note that both the Born-improved HEFT ({\tt mtdep=1}) and the \ftapprox ({\tt mtdep=2}) modes can only be run for the
SM case, as well as for variations of the trilinear Higgs coupling $c_{hhh}$. The variation of
any of the other couplings will result in an output error message and the program will exit.

The {\tt run.sh} script in the {\tt testrun} folder allows to perform 
the runs of the different stages of \POWHEG{} easily.
By typing {\tt ./run.sh} without any argument a menu with the
4 {\tt mtdep} running modes described above is shown. 
For all {\tt mtdep} running modes the code goes through all
the various steps (parallel stages) of the calculation: 
\begin{description}
 \item[{\tt parallelstage=1}:]
generation of the importance sampling grid for the Monte Carlo integration; 
 \item[{\tt parallelstage=2}:] calculation of the integral for the inclusive cross section and an upper bounding function of the integrand;
 \item[{\tt parallelstage=3}:] upper bounding factors for the generation of radiation are computed;
 \item[{\tt parallelstage=4}:] event generation, i.e. production of {\tt pwgevents-*.lhe} files.
\end{description}

\noindent Please note: if you use the script {\tt run.sh [mtdep]}, the value for {\tt mtdep} given as an argument to {\tt run.sh} will be used, even if you specified a different value for {\tt mtdep} in {\tt powheg.input-save}.

After running {\tt parallelstage=4}, the LHE files produced by \POWHEG{} can be directly showered by either \PYTHIA{} or \HERWIG{}. We provide a minimal setup for producing parton-shower matched distributions in {\tt test-pythia8}, respectively {\tt test-herwig7}. Both the angular-ordered and the dipole-shower implemented in \HERWIG{} can be operated by changing the {\tt showeralg} flag to either {\tt 'default'} or {\tt 'dipole'} in {\tt HerwigRun.sh}.

\section{Running with full top quark mass dependence ({\tt mtdep=3})}
The 2-loop virtual amplitudes in the NLO calculation with full top quark mass dependence are computed via a grid
which encodes the dependence of the virtual 2-loop amplitude on the
kinematic invariants $\hat{s}$ and
$\hat{t}$~\cite{Heinrich:2017kxx}. Please be aware that the numerical
values $m_H=125$\,GeV and $m_t=173$\,GeV are {\bf hardcoded} in this
grid and therefore should not be changed in {\tt powheg.input-save} when running in the {\tt mtdep=3} mode.
The grid is generated using python code and is directly interfaced to the
\POWHEGBOX{} fortran code via the Python/C API. In order for the grid to be found by the code, the
files ({\tt events.cdf, createdgrid.py, Virt\_full\_*E*.grid}) from the
folder {\tt Virtual} need to be copied into the local folder where the
code is run. Instead of copying the files, we suggest to create a
symbolic link to the needed files. Assuming the code is run from a
subfolder (e.g. {\tt testrun}) of the process folder, the link can be created in this subfolder as follows:
\begin{description}
\item{\tt ln -s ../Virtual/events.cdf events.cdf}
\item{\tt ln -s ../Virtual/creategrid.py creategrid.py}
\item{\tt for grid in ../Virtual/Virt\_full\_*E*.grid; do ln -s \$grid; done}
\end{description}
Once the links are in place, the code can be run with {\tt mtdep=3} as
usual. To ensure that the linked files are found, we recommend to add the run subfolder to {\tt PYTHONPATH}. 
The python code {\tt creategrid.py} will then combine the virtual grids generated with different values of the Higgs couplings
to produce a new file {\tt Virt\_full*E*.grid} corresponding to the values of the couplings chosen in the {\tt powheg.input} file.

The python code for the grid is written in python 3 and relies on the {\tt numpy} and {\tt sympy} packages, which 
the user should install separately. When building the {\tt ggHH} process the Makefile will find the embedded python 3 library
via a call to {\tt python3-config}, which the user should ensure is configured correctly and points to the correct library.
Note that on some systems the Python/C API does not search for packages (such as {\tt numpy} and {\tt sympy}) in the same 
paths as the python executable would. Therefore the user should ensure that these packages can be found also by an embedded python program.

\section{Physical input parameters}

The bottom quark is considered massless in all four {\tt mtdep} modes. The Higgs
bosons are generated on-shell with zero width. A decay can be attached
through the parton shower in the narrow-width approximation. However,
the decay is by default switched off (see the {\tt hdecaymode} flag in the
example {\tt powheg.input-save} input card in {\tt testrun}).

The masses of the Higgs boson and the top quark are set by default to
$m_h=125$\,GeV, $m_t=173$\,GeV, respectively, whereas their widths
have been set to zero. The full SM 2-loop virtual contribution has
been computed with these mass values hardcoded. 
It is no problem to change the values of $m_h$
and $m_t$ via the {\tt powheg.input-save} input card when running with
{\tt mtdep} set to $0$, $1$ or $2$.
However, when running with {\tt mtdep = 3}, 
 {\bf the values of the Higgs mass and
the top mass should be kept fixed at $m_h=125$\,GeV and $m_t=173$\,GeV}, and the widths should be zero.
Otherwise the two-loop virtual part would contain a different top or Higgs mass than the rest of the calculation.

The ratio of the Higgs couplings to their SM values, as defined in the context of the Electroweak Chiral Lagrangian (see~\cite{Buchalla:2018yce} and references within), can be varied directly in the {\tt powheg.input} card. These are, with their SM values as default:

\begin{description}
\item[{\tt chhh=1.0}:] { the ratio of the Higgs trilinear coupling to its SM value,}
\item[{\tt ct=1.0}:] { the ratio of the Higgs Yukawa coupling to the top quark to its SM value,}
\item[{\tt cthh=0.0}:] { the effective coupling of two Higgs bosons to a top quark pair,}
\item[{\tt cgg=0.0}:] { the effective contact coupling of two gluons to the Higgs boson,}
\item[{\tt cgghh=0.0}:] { the effective quartic contact coupling of two gluons to a Higgs pair.}
\end{description}

With the exception of the trilinear coupling {\tt chhh}, the Higgs couplings cannot be varied with either {\tt mtdep=1} or {\tt mtdep=2}.

The \POWHEGBOX{} offers the possibility to use a damping factor $h=\texttt{hdamp}$ of the
form~\cite{Alioli:2008tz,Alioli:2009je}
\begin{align}
  F=\frac{h^{2}}{(\pthh)^2+h^{2}}\,,
\end{align}
where \pthh is the transverse momentum of the Higgs boson pair, to
limit the amount of hard radiation which is exponentiated in the
Sudakov form factor. The default setting ($F\equiv1$, corresponding to {\tt hdamp}$=\infty$) results in
quite hard tails for observables like
$\pthh$~\cite{Heinrich:2017kxx}. Changing the damping factor $F$ by
setting the flag {\tt hdamp} to some finite value in the input
card brings the high transverse momentum tails down towards the fixed-order NLO
predictions. Varying {\tt hdamp} allows to assess shower untertainties
within the \POWHEG{} matching scheme. However, when choosing a value
for {\tt hdamp}, it is important not to cut into the Sudakov
regime. In fact, a too low value for {\tt hdamp} could spoil the
logarithmic accuracy of the prediction. For this reason we suggest not
to choose values for {\tt hdamp} below $\sim 200$. Our default value is  {\tt hdamp=250}.

\section{Trouble shooting}

Below is a list of potential issues:

\begin{itemize}
\item With {\tt mtdep=3}, the powheg *.log files contain the string ``grid not found": make sure your run folder contains the following files 
(or symbolic links to these files located in the folder `Virtual'): {\tt events.cdf, creategrid.py} and the grid files  {\tt Virt\_full\_*E*.grid}.
\item The distributions contain spikes: make sure the flags {\tt check\_bad\_st1} and {\tt check\_bad\_st2} (present in \POWHEG{} svn revisions $\geq 3600$) are set to 1. These flags discard outliers in the xgrid files. If you still find outliers with these flags switched on, the ``bad" runs should be removed 
(manually, or using the script {\tt automated\_removal.py} present in the subfolder {\tt shell}).
\item Problems related to the {\tt gcc 6.3.0} compiler have been reported. They should not occur when using a more recent {\tt gcc} compiler.
\end{itemize}


\begin{thebibliography}{10}

%\cite{Heinrich:2019bkc}
\bibitem{Heinrich:2019bkc}
  G.~Heinrich, S.~P.~Jones, M.~Kerner, G.~Luisoni and L.~Scyboz,
  ``Probing the trilinear Higgs boson coupling in di-Higgs production at NLO QCD including parton shower effects,''
  arXiv:1903.08137 [hep-ph].
  %%CITATION = ARXIV:1903.08137;%%

  %\cite{Nason:2004rx}
\bibitem{Nason:2004rx}
  P.~Nason,
  ``A new method for combining NLO QCD with shower Monte Carlo algorithms,''
  JHEP {\bf 0411} (2004) 040
  [arXiv:hep-ph/0409146].
  %%CITATION = JHEPA,0411,040;%%

  %\cite{Frixione:2007vw}
\bibitem{Frixione:2007vw}
  S.~Frixione, P.~Nason and C.~Oleari,
  ``Matching NLO QCD computations with Parton Shower simulations: the POWHEG method,''
  JHEP {\bf 0711} (2007) 070
  [arXiv:0709.2092].
  %%CITATION = JHEPA,0711,070;%%

  %\cite{Alioli:2010xd}
\bibitem{Alioli:2010xd}
  S.~Alioli, P.~Nason, C.~Oleari and E.~Re,
  ``A general framework for implementing NLO calculations in shower Monte Carlo programs: the POWHEG BOX,''
  JHEP {\bf 1006} (2010) 043
  [arXiv:1002.2581].

  %\cite{Jezo:2015aia}
\bibitem{Jezo:2015aia}
  T.~Jezo and P.~Nason,
  ``On the Treatment of Resonances in Next-to-Leading Order Calculations Matched to a Parton Shower,''
  JHEP {\bf 1512} (2015) 065
  [arXiv:1509.09071].

\bibitem{Heinrich:2017kxx}
  G.~Heinrich, S.~P.~Jones, M.~Kerner, G.~Luisoni and E.~Vryonidou,
  %``NLO predictions for Higgs boson pair production with full top quark mass dependence matched to parton showers,''
  JHEP {\bf 1708} (2017) 088
%  doi:10.1007/JHEP08(2017)088
  [arXiv:1703.09252 [hep-ph]].
  %%CITATION = doi:10.1007/JHEP08(2017)088;%%

\bibitem{Borowka:2016ehy}
  S.~Borowka, N.~Greiner, G.~Heinrich, S.~P.~Jones, M.~Kerner, J.~Schlenk, U.~Schubert and T.~Zirke,
  %``Higgs Boson Pair Production in Gluon Fusion at Next-to-Leading Order with Full Top-Quark Mass Dependence,''
  Phys.\ Rev.\ Lett.\  {\bf 117} (2016) no.1,  012001
   Erratum: [Phys.\ Rev.\ Lett.\  {\bf 117} (2016) no.7,  079901]
%  doi:10.1103/PhysRevLett.117.079901, 10.1103/PhysRevLett.117.012001
  [arXiv:1604.06447 [hep-ph]].

\bibitem{Borowka:2016ypz}
  S.~Borowka, N.~Greiner, G.~Heinrich, S.~P.~Jones, M.~Kerner, J.~Schlenk and T.~Zirke,
  %``Full top quark mass dependence in Higgs boson pair production at NLO,''
  JHEP {\bf 1610} (2016) 107
%  doi:10.1007/JHEP10(2016)107
  [arXiv:1608.04798 [hep-ph]].

%\cite{Alioli:2008tz}
\bibitem{Alioli:2008tz}
  S.~Alioli, P.~Nason, C.~Oleari and E.~Re,
  %``NLO Higgs boson production via gluon fusion matched with shower in POWHEG,''
  JHEP {\bf 0904} (2009) 002
  [arXiv:0812.0578].

  %\cite{Alioli:2009je}
\bibitem{Alioli:2009je}
  S.~Alioli, P.~Nason, C.~Oleari and E.~Re,
  ``NLO single-top production matched with shower in POWHEG: s- and t-channel contributions,''
  JHEP {\bf 0909} (2009) 111
   Erratum: [JHEP {\bf 1002} (2010) 011]
  [arXiv:0907.4076].
  
  \bibitem{Buchalla:2018yce}
  G.~Buchalla, M.~Capozi, A.~Celis, G.~Heinrich and L.~Scyboz,
  ``Higgs boson pair production in non-linear Effective Field Theory with full $m_t$-dependence at NLO QCD,''
  JHEP {\bf 1809} (2018) 057
   [APS Physics {\bf 2018} (2018) 57]
  [arXiv:1806.05162 [hep-ph]].
  
%\cite{Cullen:2014yla}
\bibitem{Cullen:2014yla}
  G.~Cullen {\it et al.},
  ``GoSam-2.0: a tool for automated one-loop calculations within the Standard Model and beyond,''
  Eur.\ Phys.\ J.\ C {\bf 74} (2014) no.8,  3001
  %doi:10.1140/epjc/s10052-014-3001-5
  [arXiv:1404.7096 [hep-ph]].

\end{thebibliography}

\end{document}





